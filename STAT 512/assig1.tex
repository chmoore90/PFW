% Options for packages loaded elsewhere
\PassOptionsToPackage{unicode}{hyperref}
\PassOptionsToPackage{hyphens}{url}
%
\documentclass[
]{article}
\usepackage{amsmath,amssymb}
\usepackage{iftex}
\ifPDFTeX
  \usepackage[T1]{fontenc}
  \usepackage[utf8]{inputenc}
  \usepackage{textcomp} % provide euro and other symbols
\else % if luatex or xetex
  \usepackage{unicode-math} % this also loads fontspec
  \defaultfontfeatures{Scale=MatchLowercase}
  \defaultfontfeatures[\rmfamily]{Ligatures=TeX,Scale=1}
\fi
\usepackage{lmodern}
\ifPDFTeX\else
  % xetex/luatex font selection
\fi
% Use upquote if available, for straight quotes in verbatim environments
\IfFileExists{upquote.sty}{\usepackage{upquote}}{}
\IfFileExists{microtype.sty}{% use microtype if available
  \usepackage[]{microtype}
  \UseMicrotypeSet[protrusion]{basicmath} % disable protrusion for tt fonts
}{}
\makeatletter
\@ifundefined{KOMAClassName}{% if non-KOMA class
  \IfFileExists{parskip.sty}{%
    \usepackage{parskip}
  }{% else
    \setlength{\parindent}{0pt}
    \setlength{\parskip}{6pt plus 2pt minus 1pt}}
}{% if KOMA class
  \KOMAoptions{parskip=half}}
\makeatother
\usepackage{xcolor}
\usepackage[margin=1in]{geometry}
\usepackage{color}
\usepackage{fancyvrb}
\newcommand{\VerbBar}{|}
\newcommand{\VERB}{\Verb[commandchars=\\\{\}]}
\DefineVerbatimEnvironment{Highlighting}{Verbatim}{commandchars=\\\{\}}
% Add ',fontsize=\small' for more characters per line
\usepackage{framed}
\definecolor{shadecolor}{RGB}{248,248,248}
\newenvironment{Shaded}{\begin{snugshade}}{\end{snugshade}}
\newcommand{\AlertTok}[1]{\textcolor[rgb]{0.94,0.16,0.16}{#1}}
\newcommand{\AnnotationTok}[1]{\textcolor[rgb]{0.56,0.35,0.01}{\textbf{\textit{#1}}}}
\newcommand{\AttributeTok}[1]{\textcolor[rgb]{0.13,0.29,0.53}{#1}}
\newcommand{\BaseNTok}[1]{\textcolor[rgb]{0.00,0.00,0.81}{#1}}
\newcommand{\BuiltInTok}[1]{#1}
\newcommand{\CharTok}[1]{\textcolor[rgb]{0.31,0.60,0.02}{#1}}
\newcommand{\CommentTok}[1]{\textcolor[rgb]{0.56,0.35,0.01}{\textit{#1}}}
\newcommand{\CommentVarTok}[1]{\textcolor[rgb]{0.56,0.35,0.01}{\textbf{\textit{#1}}}}
\newcommand{\ConstantTok}[1]{\textcolor[rgb]{0.56,0.35,0.01}{#1}}
\newcommand{\ControlFlowTok}[1]{\textcolor[rgb]{0.13,0.29,0.53}{\textbf{#1}}}
\newcommand{\DataTypeTok}[1]{\textcolor[rgb]{0.13,0.29,0.53}{#1}}
\newcommand{\DecValTok}[1]{\textcolor[rgb]{0.00,0.00,0.81}{#1}}
\newcommand{\DocumentationTok}[1]{\textcolor[rgb]{0.56,0.35,0.01}{\textbf{\textit{#1}}}}
\newcommand{\ErrorTok}[1]{\textcolor[rgb]{0.64,0.00,0.00}{\textbf{#1}}}
\newcommand{\ExtensionTok}[1]{#1}
\newcommand{\FloatTok}[1]{\textcolor[rgb]{0.00,0.00,0.81}{#1}}
\newcommand{\FunctionTok}[1]{\textcolor[rgb]{0.13,0.29,0.53}{\textbf{#1}}}
\newcommand{\ImportTok}[1]{#1}
\newcommand{\InformationTok}[1]{\textcolor[rgb]{0.56,0.35,0.01}{\textbf{\textit{#1}}}}
\newcommand{\KeywordTok}[1]{\textcolor[rgb]{0.13,0.29,0.53}{\textbf{#1}}}
\newcommand{\NormalTok}[1]{#1}
\newcommand{\OperatorTok}[1]{\textcolor[rgb]{0.81,0.36,0.00}{\textbf{#1}}}
\newcommand{\OtherTok}[1]{\textcolor[rgb]{0.56,0.35,0.01}{#1}}
\newcommand{\PreprocessorTok}[1]{\textcolor[rgb]{0.56,0.35,0.01}{\textit{#1}}}
\newcommand{\RegionMarkerTok}[1]{#1}
\newcommand{\SpecialCharTok}[1]{\textcolor[rgb]{0.81,0.36,0.00}{\textbf{#1}}}
\newcommand{\SpecialStringTok}[1]{\textcolor[rgb]{0.31,0.60,0.02}{#1}}
\newcommand{\StringTok}[1]{\textcolor[rgb]{0.31,0.60,0.02}{#1}}
\newcommand{\VariableTok}[1]{\textcolor[rgb]{0.00,0.00,0.00}{#1}}
\newcommand{\VerbatimStringTok}[1]{\textcolor[rgb]{0.31,0.60,0.02}{#1}}
\newcommand{\WarningTok}[1]{\textcolor[rgb]{0.56,0.35,0.01}{\textbf{\textit{#1}}}}
\usepackage{graphicx}
\makeatletter
\def\maxwidth{\ifdim\Gin@nat@width>\linewidth\linewidth\else\Gin@nat@width\fi}
\def\maxheight{\ifdim\Gin@nat@height>\textheight\textheight\else\Gin@nat@height\fi}
\makeatother
% Scale images if necessary, so that they will not overflow the page
% margins by default, and it is still possible to overwrite the defaults
% using explicit options in \includegraphics[width, height, ...]{}
\setkeys{Gin}{width=\maxwidth,height=\maxheight,keepaspectratio}
% Set default figure placement to htbp
\makeatletter
\def\fps@figure{htbp}
\makeatother
\setlength{\emergencystretch}{3em} % prevent overfull lines
\providecommand{\tightlist}{%
  \setlength{\itemsep}{0pt}\setlength{\parskip}{0pt}}
\setcounter{secnumdepth}{-\maxdimen} % remove section numbering
\ifLuaTeX
  \usepackage{selnolig}  % disable illegal ligatures
\fi
\IfFileExists{bookmark.sty}{\usepackage{bookmark}}{\usepackage{hyperref}}
\IfFileExists{xurl.sty}{\usepackage{xurl}}{} % add URL line breaks if available
\urlstyle{same}
\hypersetup{
  pdftitle={Question 8},
  pdfauthor={Chloe Moore},
  hidelinks,
  pdfcreator={LaTeX via pandoc}}

\title{Question 8}
\author{Chloe Moore}
\date{2024-09-11}

\begin{document}
\maketitle

\hypertarget{part-ai}{%
\paragraph{Part a(i)}\label{part-ai}}

\begin{Shaded}
\begin{Highlighting}[]
\NormalTok{data }\OtherTok{=} \FunctionTok{read.csv}\NormalTok{(}\StringTok{"WiscNursingHome.csv"}\NormalTok{)}

\FunctionTok{cor}\NormalTok{(data}\SpecialCharTok{$}\NormalTok{TPY, }\FunctionTok{log}\NormalTok{(data}\SpecialCharTok{$}\NormalTok{TPY))}
\end{Highlighting}
\end{Shaded}

\begin{verbatim}
## [1] 0.9288011
\end{verbatim}

The correlation value is quite close to 1, which means that TPY and it's
log value are strongly positively correlated.

\hypertarget{part-aii}{%
\paragraph{Part a(ii)}\label{part-aii}}

\begin{Shaded}
\begin{Highlighting}[]
\FunctionTok{cor}\NormalTok{(data[}\FunctionTok{c}\NormalTok{(}\StringTok{"TPY"}\NormalTok{, }\StringTok{"NUMBED"}\NormalTok{, }\StringTok{"SQRFOOT"}\NormalTok{)], }\AttributeTok{use =} \StringTok{"complete.obs"}\NormalTok{)}
\end{Highlighting}
\end{Shaded}

\begin{verbatim}
##               TPY    NUMBED   SQRFOOT
## TPY     1.0000000 0.9836241 0.8219443
## NUMBED  0.9836241 1.0000000 0.8136223
## SQRFOOT 0.8219443 0.8136223 1.0000000
\end{verbatim}

All the correlations we see here are close to 1. This means TPY, NUMBED,
and SQRFOOT are all positively correlated with each other.

\hypertarget{part-aiii}{%
\paragraph{Part a(iii)}\label{part-aiii}}

\begin{Shaded}
\begin{Highlighting}[]
\FunctionTok{cor}\NormalTok{(data}\SpecialCharTok{$}\NormalTok{TPY, data}\SpecialCharTok{$}\NormalTok{NUMBED}\SpecialCharTok{/}\DecValTok{10}\NormalTok{)}
\end{Highlighting}
\end{Shaded}

\begin{verbatim}
## [1] 0.9837719
\end{verbatim}

This number is very close the the original correlation value for TPY and
NUMBED (0.9836241). This could indicate that the positive correlation
between these variables remains strong, regardless of the value of
NUMBED.

\hypertarget{part-b}{%
\paragraph{Part b)}\label{part-b}}

\includegraphics{assig1_files/figure-latex/unnamed-chunk-4-1.pdf}
\includegraphics{assig1_files/figure-latex/unnamed-chunk-4-2.pdf} The
TPY and NUMBED plot has the data points clustered closely together along
a linear trend line. This suggest a strong correlation between these
variables, and this is supported by the correlation value we found
earlier. The TPY and SQRFOOT plot also showed a linear trend, but the
data was more scattered. This increased variance suggests the
correlation between TPY and SQRFOOT is not as strong (but still
definitely present!) as NUMBED. This is supported by the correlation
values we found earlier too.

\hypertarget{part-ci}{%
\paragraph{Part c(i)}\label{part-ci}}

\begin{Shaded}
\begin{Highlighting}[]
\NormalTok{fit1 }\OtherTok{=} \FunctionTok{lm}\NormalTok{(TPY }\SpecialCharTok{\textasciitilde{}}\NormalTok{ NUMBED, data)}
\FunctionTok{summary}\NormalTok{(fit1)}
\end{Highlighting}
\end{Shaded}

\begin{verbatim}
## 
## Call:
## lm(formula = TPY ~ NUMBED, data = data)
## 
## Residuals:
##     Min      1Q  Median      3Q     Max 
## -63.249  -2.067   0.866   3.967  37.064 
## 
## Coefficients:
##              Estimate Std. Error t value Pr(>|t|)    
## (Intercept) -0.877827   0.692503  -1.268    0.205    
## NUMBED       0.927191   0.006324 146.611   <2e-16 ***
## ---
## Signif. codes:  0 '***' 0.001 '**' 0.01 '*' 0.05 '.' 0.1 ' ' 1
## 
## Residual standard error: 8.538 on 715 degrees of freedom
## Multiple R-squared:  0.9678, Adjusted R-squared:  0.9678 
## F-statistic: 2.149e+04 on 1 and 715 DF,  p-value: < 2.2e-16
\end{verbatim}

R\^{}2 = 0.9678, t-value = 146.611. These results suggest a very good
fit, since the R\^{}2 is close to 1 and t-value is relatively high.

\hypertarget{part-cii}{%
\paragraph{Part c(ii)}\label{part-cii}}

\begin{Shaded}
\begin{Highlighting}[]
\NormalTok{fit2 }\OtherTok{=} \FunctionTok{lm}\NormalTok{(TPY }\SpecialCharTok{\textasciitilde{}}\NormalTok{ SQRFOOT, data)}
\FunctionTok{summary}\NormalTok{(fit2)}
\end{Highlighting}
\end{Shaded}

\begin{verbatim}
## 
## Call:
## lm(formula = TPY ~ SQRFOOT, data = data)
## 
## Residuals:
##      Min       1Q   Median       3Q      Max 
## -114.501  -15.391   -2.426   15.615  126.599 
## 
## Coefficients:
##             Estimate Std. Error t value Pr(>|t|)    
## (Intercept) 33.54754    1.78645   18.78   <2e-16 ***
## SQRFOOT      1.11786    0.02917   38.32   <2e-16 ***
## ---
## Signif. codes:  0 '***' 0.001 '**' 0.01 '*' 0.05 '.' 0.1 ' ' 1
## 
## Residual standard error: 27.14 on 705 degrees of freedom
##   (10 observations deleted due to missingness)
## Multiple R-squared:  0.6756, Adjusted R-squared:  0.6751 
## F-statistic:  1468 on 1 and 705 DF,  p-value: < 2.2e-16
\end{verbatim}

R\^{}2 = 0.6756, t-value = 38.32. The model using NUMBED fit better, as
it had a R\^{}2 score closer to 1.

\hypertarget{part-ciii}{%
\paragraph{Part c(iii)}\label{part-ciii}}

\begin{Shaded}
\begin{Highlighting}[]
\NormalTok{fit3 }\OtherTok{=} \FunctionTok{lm}\NormalTok{(}\FunctionTok{log}\NormalTok{(TPY) }\SpecialCharTok{\textasciitilde{}} \FunctionTok{log}\NormalTok{(NUMBED), data)}
\FunctionTok{summary}\NormalTok{(fit3)}
\end{Highlighting}
\end{Shaded}

\begin{verbatim}
## 
## Call:
## lm(formula = log(TPY) ~ log(NUMBED), data = data)
## 
## Residuals:
##      Min       1Q   Median       3Q      Max 
## -0.89233 -0.01923  0.01741  0.05768  0.29396 
## 
## Coefficients:
##              Estimate Std. Error t value Pr(>|t|)    
## (Intercept) -0.163315   0.036045  -4.531 6.88e-06 ***
## log(NUMBED)  1.015739   0.008038 126.372  < 2e-16 ***
## ---
## Signif. codes:  0 '***' 0.001 '**' 0.01 '*' 0.05 '.' 0.1 ' ' 1
## 
## Residual standard error: 0.1058 on 715 degrees of freedom
## Multiple R-squared:  0.9571, Adjusted R-squared:  0.9571 
## F-statistic: 1.597e+04 on 1 and 715 DF,  p-value: < 2.2e-16
\end{verbatim}

\hypertarget{part-civ}{%
\paragraph{Part c(iv)}\label{part-civ}}

\begin{Shaded}
\begin{Highlighting}[]
\NormalTok{fit4 }\OtherTok{=} \FunctionTok{lm}\NormalTok{(}\FunctionTok{log}\NormalTok{(TPY) }\SpecialCharTok{\textasciitilde{}} \FunctionTok{log}\NormalTok{(SQRFOOT), data)}
\FunctionTok{summary}\NormalTok{(fit4)}
\end{Highlighting}
\end{Shaded}

\begin{verbatim}
## 
## Call:
## lm(formula = log(TPY) ~ log(SQRFOOT), data = data)
## 
## Residuals:
##      Min       1Q   Median       3Q      Max 
## -1.73615 -0.15269  0.02281  0.19078  1.06551 
## 
## Coefficients:
##              Estimate Std. Error t value Pr(>|t|)    
## (Intercept)   1.83032    0.06797   26.93   <2e-16 ***
## log(SQRFOOT)  0.68163    0.01800   37.87   <2e-16 ***
## ---
## Signif. codes:  0 '***' 0.001 '**' 0.01 '*' 0.05 '.' 0.1 ' ' 1
## 
## Residual standard error: 0.2925 on 705 degrees of freedom
##   (10 observations deleted due to missingness)
## Multiple R-squared:  0.6704, Adjusted R-squared:   0.67 
## F-statistic:  1434 on 1 and 705 DF,  p-value: < 2.2e-16
\end{verbatim}

\end{document}
