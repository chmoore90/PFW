% Options for packages loaded elsewhere
\PassOptionsToPackage{unicode}{hyperref}
\PassOptionsToPackage{hyphens}{url}
%
\documentclass[
]{article}
\usepackage{amsmath,amssymb}
\usepackage{iftex}
\ifPDFTeX
  \usepackage[T1]{fontenc}
  \usepackage[utf8]{inputenc}
  \usepackage{textcomp} % provide euro and other symbols
\else % if luatex or xetex
  \usepackage{unicode-math} % this also loads fontspec
  \defaultfontfeatures{Scale=MatchLowercase}
  \defaultfontfeatures[\rmfamily]{Ligatures=TeX,Scale=1}
\fi
\usepackage{lmodern}
\ifPDFTeX\else
  % xetex/luatex font selection
\fi
% Use upquote if available, for straight quotes in verbatim environments
\IfFileExists{upquote.sty}{\usepackage{upquote}}{}
\IfFileExists{microtype.sty}{% use microtype if available
  \usepackage[]{microtype}
  \UseMicrotypeSet[protrusion]{basicmath} % disable protrusion for tt fonts
}{}
\makeatletter
\@ifundefined{KOMAClassName}{% if non-KOMA class
  \IfFileExists{parskip.sty}{%
    \usepackage{parskip}
  }{% else
    \setlength{\parindent}{0pt}
    \setlength{\parskip}{6pt plus 2pt minus 1pt}}
}{% if KOMA class
  \KOMAoptions{parskip=half}}
\makeatother
\usepackage{xcolor}
\usepackage[margin=1in]{geometry}
\usepackage{color}
\usepackage{fancyvrb}
\newcommand{\VerbBar}{|}
\newcommand{\VERB}{\Verb[commandchars=\\\{\}]}
\DefineVerbatimEnvironment{Highlighting}{Verbatim}{commandchars=\\\{\}}
% Add ',fontsize=\small' for more characters per line
\usepackage{framed}
\definecolor{shadecolor}{RGB}{248,248,248}
\newenvironment{Shaded}{\begin{snugshade}}{\end{snugshade}}
\newcommand{\AlertTok}[1]{\textcolor[rgb]{0.94,0.16,0.16}{#1}}
\newcommand{\AnnotationTok}[1]{\textcolor[rgb]{0.56,0.35,0.01}{\textbf{\textit{#1}}}}
\newcommand{\AttributeTok}[1]{\textcolor[rgb]{0.13,0.29,0.53}{#1}}
\newcommand{\BaseNTok}[1]{\textcolor[rgb]{0.00,0.00,0.81}{#1}}
\newcommand{\BuiltInTok}[1]{#1}
\newcommand{\CharTok}[1]{\textcolor[rgb]{0.31,0.60,0.02}{#1}}
\newcommand{\CommentTok}[1]{\textcolor[rgb]{0.56,0.35,0.01}{\textit{#1}}}
\newcommand{\CommentVarTok}[1]{\textcolor[rgb]{0.56,0.35,0.01}{\textbf{\textit{#1}}}}
\newcommand{\ConstantTok}[1]{\textcolor[rgb]{0.56,0.35,0.01}{#1}}
\newcommand{\ControlFlowTok}[1]{\textcolor[rgb]{0.13,0.29,0.53}{\textbf{#1}}}
\newcommand{\DataTypeTok}[1]{\textcolor[rgb]{0.13,0.29,0.53}{#1}}
\newcommand{\DecValTok}[1]{\textcolor[rgb]{0.00,0.00,0.81}{#1}}
\newcommand{\DocumentationTok}[1]{\textcolor[rgb]{0.56,0.35,0.01}{\textbf{\textit{#1}}}}
\newcommand{\ErrorTok}[1]{\textcolor[rgb]{0.64,0.00,0.00}{\textbf{#1}}}
\newcommand{\ExtensionTok}[1]{#1}
\newcommand{\FloatTok}[1]{\textcolor[rgb]{0.00,0.00,0.81}{#1}}
\newcommand{\FunctionTok}[1]{\textcolor[rgb]{0.13,0.29,0.53}{\textbf{#1}}}
\newcommand{\ImportTok}[1]{#1}
\newcommand{\InformationTok}[1]{\textcolor[rgb]{0.56,0.35,0.01}{\textbf{\textit{#1}}}}
\newcommand{\KeywordTok}[1]{\textcolor[rgb]{0.13,0.29,0.53}{\textbf{#1}}}
\newcommand{\NormalTok}[1]{#1}
\newcommand{\OperatorTok}[1]{\textcolor[rgb]{0.81,0.36,0.00}{\textbf{#1}}}
\newcommand{\OtherTok}[1]{\textcolor[rgb]{0.56,0.35,0.01}{#1}}
\newcommand{\PreprocessorTok}[1]{\textcolor[rgb]{0.56,0.35,0.01}{\textit{#1}}}
\newcommand{\RegionMarkerTok}[1]{#1}
\newcommand{\SpecialCharTok}[1]{\textcolor[rgb]{0.81,0.36,0.00}{\textbf{#1}}}
\newcommand{\SpecialStringTok}[1]{\textcolor[rgb]{0.31,0.60,0.02}{#1}}
\newcommand{\StringTok}[1]{\textcolor[rgb]{0.31,0.60,0.02}{#1}}
\newcommand{\VariableTok}[1]{\textcolor[rgb]{0.00,0.00,0.00}{#1}}
\newcommand{\VerbatimStringTok}[1]{\textcolor[rgb]{0.31,0.60,0.02}{#1}}
\newcommand{\WarningTok}[1]{\textcolor[rgb]{0.56,0.35,0.01}{\textbf{\textit{#1}}}}
\usepackage{graphicx}
\makeatletter
\def\maxwidth{\ifdim\Gin@nat@width>\linewidth\linewidth\else\Gin@nat@width\fi}
\def\maxheight{\ifdim\Gin@nat@height>\textheight\textheight\else\Gin@nat@height\fi}
\makeatother
% Scale images if necessary, so that they will not overflow the page
% margins by default, and it is still possible to overwrite the defaults
% using explicit options in \includegraphics[width, height, ...]{}
\setkeys{Gin}{width=\maxwidth,height=\maxheight,keepaspectratio}
% Set default figure placement to htbp
\makeatletter
\def\fps@figure{htbp}
\makeatother
\setlength{\emergencystretch}{3em} % prevent overfull lines
\providecommand{\tightlist}{%
  \setlength{\itemsep}{0pt}\setlength{\parskip}{0pt}}
\setcounter{secnumdepth}{-\maxdimen} % remove section numbering
\ifLuaTeX
  \usepackage{selnolig}  % disable illegal ligatures
\fi
\IfFileExists{bookmark.sty}{\usepackage{bookmark}}{\usepackage{hyperref}}
\IfFileExists{xurl.sty}{\usepackage{xurl}}{} % add URL line breaks if available
\urlstyle{same}
\hypersetup{
  pdftitle={STAT 531 Assignment 1},
  pdfauthor={Chloe Moore},
  hidelinks,
  pdfcreator={LaTeX via pandoc}}

\title{STAT 531 Assignment 1}
\author{Chloe Moore}
\date{2024-09-27}

\begin{document}
\maketitle

\begin{Shaded}
\begin{Highlighting}[]
\FunctionTok{library}\NormalTok{(ggplot2)}
\NormalTok{data }\OtherTok{=} \FunctionTok{read.csv}\NormalTok{(}\StringTok{"Loandefault.csv"}\NormalTok{, }\AttributeTok{header =} \ConstantTok{TRUE}\NormalTok{)}
\FunctionTok{attach}\NormalTok{(data)}
\end{Highlighting}
\end{Shaded}

\hypertarget{question-1}{%
\subsubsection{Question 1:}\label{question-1}}

\hypertarget{amt_income_total}{%
\paragraph{AMT\_INCOME\_TOTAL}\label{amt_income_total}}

\begin{Shaded}
\begin{Highlighting}[]
\FunctionTok{ggplot}\NormalTok{(data, }\FunctionTok{aes}\NormalTok{(}\AttributeTok{x=}\NormalTok{AMT\_INCOME\_TOTAL)) }\SpecialCharTok{+}
  \FunctionTok{geom\_histogram}\NormalTok{(}\AttributeTok{bins=}\DecValTok{20}\NormalTok{, }\AttributeTok{fill=}\StringTok{"steelblue"}\NormalTok{, }\AttributeTok{color=}\StringTok{"black"}\NormalTok{)}
\end{Highlighting}
\end{Shaded}

\includegraphics{Assig1_files/figure-latex/unnamed-chunk-2-1.pdf}

\begin{Shaded}
\begin{Highlighting}[]
\FunctionTok{summary}\NormalTok{(AMT\_INCOME\_TOTAL)}
\end{Highlighting}
\end{Shaded}

\begin{verbatim}
##      Min.   1st Qu.    Median      Mean   3rd Qu.      Max. 
##     25650    112500    147150    168798    202500 117000000
\end{verbatim}

\hypertarget{response-part-1}{%
\paragraph{Response (part 1):}\label{response-part-1}}

This distribution is extremely skewed right. The range between the
maximum and minimum values is also huge. The right skew indicates there
are not many outlying high values, as otherwise their bars would show up
on the histogram. Since there are so few of these high outliers, I think
we can safely eliminate them. To do this, while still preserving as much
data as possible, I removed the top 2.5\% of the values. In statistical
studies, it is common to use an alpha level of 0.05, where the highest
and lowest 2.5\% of the distribution are considered significant. I am
simply following this precedent, except I have elected to keep the lower
2.5\% because the right skew indicates the data on this side of the
graph that may not be outliers (at least not as extreme).

\begin{Shaded}
\begin{Highlighting}[]
\NormalTok{upper\_bound }\OtherTok{=} \FunctionTok{quantile}\NormalTok{(AMT\_INCOME\_TOTAL, }\FloatTok{0.975}\NormalTok{)}
\NormalTok{data\_capped }\OtherTok{=} \FunctionTok{subset}\NormalTok{(data, AMT\_INCOME\_TOTAL }\SpecialCharTok{\textless{}=}\NormalTok{ upper\_bound)}

\FunctionTok{ggplot}\NormalTok{(data\_capped, }\FunctionTok{aes}\NormalTok{(}\AttributeTok{x=}\NormalTok{AMT\_INCOME\_TOTAL)) }\SpecialCharTok{+}
  \FunctionTok{geom\_histogram}\NormalTok{(}\AttributeTok{bins=}\DecValTok{20}\NormalTok{, }\AttributeTok{fill=}\StringTok{"steelblue"}\NormalTok{, }\AttributeTok{color=}\StringTok{"black"}\NormalTok{)}
\end{Highlighting}
\end{Shaded}

\includegraphics{Assig1_files/figure-latex/unnamed-chunk-3-1.pdf}

\begin{Shaded}
\begin{Highlighting}[]
\FunctionTok{summary}\NormalTok{(data\_capped}\SpecialCharTok{$}\NormalTok{AMT\_INCOME\_TOTAL)}
\end{Highlighting}
\end{Shaded}

\begin{verbatim}
##    Min. 1st Qu.  Median    Mean 3rd Qu.    Max. 
##   25650  112500  144000  159861  202500  405000
\end{verbatim}

\hypertarget{response-part-2}{%
\paragraph{Response (part 2):}\label{response-part-2}}

The new graph is still skewed right, but now we can actually see where
the majority of our values lie. The distribution has one mode and the
values cluster around 150,000. We can see the affect the outliers were
having on the mean, as it is much lower now than when the outliers were
included.

\hypertarget{were-going-to-keep-using-the-capped-data-set-for-the-rest-of-the-assignment.}{%
\subsubsection{We're going to keep using the capped data set for the
rest of the
assignment.}\label{were-going-to-keep-using-the-capped-data-set-for-the-rest-of-the-assignment.}}

\begin{Shaded}
\begin{Highlighting}[]
\FunctionTok{detach}\NormalTok{(data)}
\FunctionTok{attach}\NormalTok{(data\_capped)}
\end{Highlighting}
\end{Shaded}

\hypertarget{amt_credit}{%
\paragraph{AMT\_CREDIT}\label{amt_credit}}

\begin{Shaded}
\begin{Highlighting}[]
\FunctionTok{ggplot}\NormalTok{(data\_capped, }\FunctionTok{aes}\NormalTok{(}\AttributeTok{x=}\NormalTok{AMT\_CREDIT)) }\SpecialCharTok{+}
  \FunctionTok{geom\_histogram}\NormalTok{(}\AttributeTok{bins=}\DecValTok{20}\NormalTok{, }\AttributeTok{fill=}\StringTok{"orange"}\NormalTok{, }\AttributeTok{color=}\StringTok{"black"}\NormalTok{)}
\end{Highlighting}
\end{Shaded}

\includegraphics{Assig1_files/figure-latex/unnamed-chunk-5-1.pdf}

\begin{Shaded}
\begin{Highlighting}[]
\FunctionTok{summary}\NormalTok{(AMT\_CREDIT)}
\end{Highlighting}
\end{Shaded}

\begin{verbatim}
##    Min. 1st Qu.  Median    Mean 3rd Qu.    Max. 
##   45000  270000  508496  589225  808650 4050000
\end{verbatim}

\hypertarget{response}{%
\paragraph{Response:}\label{response}}

Even on our new capped data set, this distribution is skewed right as
well. All of its data are clustered on the left side of the chart, so
the mode is one our lowest values. I considered cutting off the top
2.5\% of this variable too, but ultimately I decided not to. While the
distribution is heavily skewed, I think the data are spread out enough
that we can work with it.

\hypertarget{cnt_children}{%
\paragraph{CNT\_CHILDREN}\label{cnt_children}}

\begin{Shaded}
\begin{Highlighting}[]
\FunctionTok{ggplot}\NormalTok{(data\_capped, }\FunctionTok{aes}\NormalTok{(}\AttributeTok{x=}\NormalTok{CNT\_CHILDREN)) }\SpecialCharTok{+}
  \FunctionTok{geom\_histogram}\NormalTok{(}\AttributeTok{bins=}\DecValTok{20}\NormalTok{, }\AttributeTok{fill=}\StringTok{"forestgreen"}\NormalTok{, }\AttributeTok{color=}\StringTok{"black"}\NormalTok{)}
\end{Highlighting}
\end{Shaded}

\includegraphics{Assig1_files/figure-latex/unnamed-chunk-6-1.pdf}

\begin{Shaded}
\begin{Highlighting}[]
\FunctionTok{summary}\NormalTok{(CNT\_CHILDREN)}
\end{Highlighting}
\end{Shaded}

\begin{verbatim}
##    Min. 1st Qu.  Median    Mean 3rd Qu.    Max. 
##  0.0000  0.0000  0.0000  0.4154  1.0000 19.0000
\end{verbatim}

\begin{Shaded}
\begin{Highlighting}[]
\NormalTok{upper\_bound2 }\OtherTok{=} \FunctionTok{quantile}\NormalTok{(CNT\_CHILDREN, }\FloatTok{0.999}\NormalTok{)}
\NormalTok{data\_capped2 }\OtherTok{=} \FunctionTok{subset}\NormalTok{(data\_capped, CNT\_CHILDREN }\SpecialCharTok{\textless{}=}\NormalTok{ upper\_bound2)}
\FunctionTok{ggplot}\NormalTok{(data\_capped2, }\FunctionTok{aes}\NormalTok{(}\AttributeTok{x=}\NormalTok{CNT\_CHILDREN)) }\SpecialCharTok{+}
  \FunctionTok{geom\_histogram}\NormalTok{(}\AttributeTok{bins=}\DecValTok{10}\NormalTok{, }\AttributeTok{fill=}\StringTok{"forestgreen"}\NormalTok{, }\AttributeTok{color=}\StringTok{"black"}\NormalTok{)}
\end{Highlighting}
\end{Shaded}

\includegraphics{Assig1_files/figure-latex/unnamed-chunk-7-1.pdf}

\begin{Shaded}
\begin{Highlighting}[]
\FunctionTok{summary}\NormalTok{(data\_capped2}\SpecialCharTok{$}\NormalTok{CNT\_CHILDREN)}
\end{Highlighting}
\end{Shaded}

\begin{verbatim}
##    Min. 1st Qu.  Median    Mean 3rd Qu.    Max. 
##  0.0000  0.0000  0.0000  0.4131  1.0000  4.0000
\end{verbatim}

\hypertarget{response-1}{%
\paragraph{Response:}\label{response-1}}

Here we have a similar case to our AMT\_INCOME\_TOTAL, although not
quite as dramatic. I still wanted to remove outliers so our histogram
was more than a couple bars and a very long tail. Since the range of
unique values for this variable is comparatively tiny (only 0-19), I had
to be careful not to set the cap too low. I chose to remove only the top
0.1\%.

\n\n Even though we kept most of the data, we dramatically reduced the
range of unique values for this variable. Like before, the outliers we
removed were so few in number that dropping them shouldn't hurt our
analysis. This data resembles the shape of our AMT\_CREDIT distribution.
It is heavily skewed right and the mode is, in this case, our smallest
value.

\hypertarget{were-going-to-keep-using-this-new-capped-data-set-for-the-rest-of-the-assignment.}{%
\subsubsection{We're going to keep using this NEW capped data set for
the rest of the
assignment.}\label{were-going-to-keep-using-this-new-capped-data-set-for-the-rest-of-the-assignment.}}

\begin{Shaded}
\begin{Highlighting}[]
\FunctionTok{detach}\NormalTok{(data\_capped)}
\FunctionTok{attach}\NormalTok{(data\_capped2)}
\end{Highlighting}
\end{Shaded}

\hypertarget{question-2}{%
\subsubsection{Question 2}\label{question-2}}

\hypertarget{amt_income_total-1}{%
\paragraph{AMT\_INCOME\_TOTAL}\label{amt_income_total-1}}

\begin{Shaded}
\begin{Highlighting}[]
\FunctionTok{ggplot}\NormalTok{(data\_capped2, }\FunctionTok{aes}\NormalTok{(}\AttributeTok{y=}\NormalTok{AMT\_INCOME\_TOTAL, }\AttributeTok{x=}\FunctionTok{factor}\NormalTok{(TARGET))) }\SpecialCharTok{+}
  \FunctionTok{geom\_boxplot}\NormalTok{() }\SpecialCharTok{+}
  \FunctionTok{labs}\NormalTok{(}\AttributeTok{x=}\StringTok{"FINANCIAL DIFFICULTY?"}\NormalTok{) }\SpecialCharTok{+}
  \FunctionTok{scale\_x\_discrete}\NormalTok{(}\AttributeTok{label=}\FunctionTok{c}\NormalTok{(}\StringTok{"False"}\NormalTok{, }\StringTok{"True"}\NormalTok{))}
\end{Highlighting}
\end{Shaded}

\includegraphics{Assig1_files/figure-latex/unnamed-chunk-9-1.pdf}

\begin{Shaded}
\begin{Highlighting}[]
\FunctionTok{summary}\NormalTok{(AMT\_INCOME\_TOTAL[TARGET}\SpecialCharTok{==}\DecValTok{1}\NormalTok{]) }\CommentTok{\# True column}
\end{Highlighting}
\end{Shaded}

\begin{verbatim}
##    Min. 1st Qu.  Median    Mean 3rd Qu.    Max. 
##   25650  112500  135000  155108  189000  405000
\end{verbatim}

\begin{Shaded}
\begin{Highlighting}[]
\FunctionTok{summary}\NormalTok{(AMT\_INCOME\_TOTAL[TARGET}\SpecialCharTok{==}\DecValTok{0}\NormalTok{]) }\CommentTok{\# False column}
\end{Highlighting}
\end{Shaded}

\begin{verbatim}
##    Min. 1st Qu.  Median    Mean 3rd Qu.    Max. 
##   25650  112500  144000  160282  202500  405000
\end{verbatim}

\hypertarget{amt_credit-1}{%
\paragraph{AMT\_CREDIT}\label{amt_credit-1}}

\begin{Shaded}
\begin{Highlighting}[]
\FunctionTok{ggplot}\NormalTok{(data\_capped2, }\FunctionTok{aes}\NormalTok{(}\AttributeTok{y=}\NormalTok{AMT\_CREDIT, }\AttributeTok{x=}\FunctionTok{factor}\NormalTok{(TARGET))) }\SpecialCharTok{+}
  \FunctionTok{geom\_boxplot}\NormalTok{() }\SpecialCharTok{+}
  \FunctionTok{labs}\NormalTok{(}\AttributeTok{x=}\StringTok{"FINANCIAL DIFFICULTY?"}\NormalTok{) }\SpecialCharTok{+}
  \FunctionTok{scale\_x\_discrete}\NormalTok{(}\AttributeTok{label=}\FunctionTok{c}\NormalTok{(}\StringTok{"False"}\NormalTok{, }\StringTok{"True"}\NormalTok{))}
\end{Highlighting}
\end{Shaded}

\includegraphics{Assig1_files/figure-latex/unnamed-chunk-10-1.pdf}

\begin{Shaded}
\begin{Highlighting}[]
\FunctionTok{summary}\NormalTok{(AMT\_CREDIT[TARGET}\SpecialCharTok{==}\DecValTok{1}\NormalTok{]) }\CommentTok{\# True}
\end{Highlighting}
\end{Shaded}

\begin{verbatim}
##    Min. 1st Qu.  Median    Mean 3rd Qu.    Max. 
##   45000  284400  495000  552503  728460 3020760
\end{verbatim}

\begin{Shaded}
\begin{Highlighting}[]
\FunctionTok{summary}\NormalTok{(AMT\_CREDIT[TARGET}\SpecialCharTok{==}\DecValTok{0}\NormalTok{]) }\CommentTok{\# False}
\end{Highlighting}
\end{Shaded}

\begin{verbatim}
##    Min. 1st Qu.  Median    Mean 3rd Qu.    Max. 
##   45000  270000  508496  592490  808650 4050000
\end{verbatim}

\hypertarget{cnt_children-1}{%
\paragraph{CNT\_CHILDREN}\label{cnt_children-1}}

\begin{Shaded}
\begin{Highlighting}[]
\FunctionTok{ggplot}\NormalTok{(data\_capped2, }\FunctionTok{aes}\NormalTok{(}\AttributeTok{y=}\NormalTok{CNT\_CHILDREN, }\AttributeTok{x=}\FunctionTok{factor}\NormalTok{(TARGET))) }\SpecialCharTok{+}
  \FunctionTok{geom\_boxplot}\NormalTok{() }\SpecialCharTok{+}
  \FunctionTok{labs}\NormalTok{(}\AttributeTok{x=}\StringTok{"FINANCIAL DIFFICULTY?"}\NormalTok{) }\SpecialCharTok{+}
  \FunctionTok{scale\_x\_discrete}\NormalTok{(}\AttributeTok{label=}\FunctionTok{c}\NormalTok{(}\StringTok{"False"}\NormalTok{, }\StringTok{"True"}\NormalTok{))}
\end{Highlighting}
\end{Shaded}

\includegraphics{Assig1_files/figure-latex/unnamed-chunk-11-1.pdf}

\begin{Shaded}
\begin{Highlighting}[]
\FunctionTok{summary}\NormalTok{(CNT\_CHILDREN[TARGET}\SpecialCharTok{==}\DecValTok{1}\NormalTok{]) }\CommentTok{\# True}
\end{Highlighting}
\end{Shaded}

\begin{verbatim}
##    Min. 1st Qu.  Median    Mean 3rd Qu.    Max. 
##  0.0000  0.0000  0.0000  0.4593  1.0000  4.0000
\end{verbatim}

\begin{Shaded}
\begin{Highlighting}[]
\FunctionTok{summary}\NormalTok{(CNT\_CHILDREN[TARGET}\SpecialCharTok{==}\DecValTok{0}\NormalTok{]) }\CommentTok{\# False}
\end{Highlighting}
\end{Shaded}

\begin{verbatim}
##    Min. 1st Qu.  Median    Mean 3rd Qu.    Max. 
##   0.000   0.000   0.000   0.409   1.000   4.000
\end{verbatim}

\hypertarget{response-part-1-1}{%
\paragraph{Response (part 1):}\label{response-part-1-1}}

For all three of our variables, the boxplots appear very similar. Their
quantile values and even their outliers are relatively close. Although
it's true that the data for customers with a history of financial
difficulty (True column) have a mean, median, and quantile values that
are slightly lower than the other customers, the difference is slight
and I'm not prepared to make any conclusions based on it. We'd need to
do more analysis determine that. For now, it looks like total income,
credit and number of children do not have a significant impact on
whether or not customers have a history of financial difficulty.

\begin{Shaded}
\begin{Highlighting}[]
\FunctionTok{ggplot}\NormalTok{(data\_capped, }\FunctionTok{aes}\NormalTok{(}\AttributeTok{y=}\NormalTok{CNT\_CHILDREN, }\AttributeTok{x=}\FunctionTok{factor}\NormalTok{(TARGET))) }\SpecialCharTok{+}
  \FunctionTok{geom\_boxplot}\NormalTok{() }\SpecialCharTok{+}
  \FunctionTok{labs}\NormalTok{(}\AttributeTok{x=}\StringTok{"FINANCIAL DIFFICULTY?"}\NormalTok{) }\SpecialCharTok{+}
  \FunctionTok{scale\_x\_discrete}\NormalTok{(}\AttributeTok{label=}\FunctionTok{c}\NormalTok{(}\StringTok{"False"}\NormalTok{, }\StringTok{"True"}\NormalTok{))}
\end{Highlighting}
\end{Shaded}

\includegraphics{Assig1_files/figure-latex/unnamed-chunk-12-1.pdf}

\hypertarget{responsepart-2}{%
\paragraph{Response(part 2):}\label{responsepart-2}}

Since the boxplots for CNT\_CHILDREN were so similar, I also made a box
plot for that variable using its full range (0-19). It possible that the
impact of having children does not appear until the count it much
greater than average. I was surprised to find there wasn't a large
difference with this data either. If anything, there were more data
points for families with greater numbers of children in the ``False''
column than there were in the True'' one! This tells me that number of
children mostly likely doesn't have an impact of whether or not the
client had past financial difficulties.

\hypertarget{question-3}{%
\subsubsection{Question 3}\label{question-3}}

\hypertarget{own_realty}{%
\paragraph{OWN\_REALTY}\label{own_realty}}

\begin{Shaded}
\begin{Highlighting}[]
\FunctionTok{ggplot}\NormalTok{(data\_capped2, }\FunctionTok{aes}\NormalTok{(}\AttributeTok{x=}\FunctionTok{factor}\NormalTok{(TARGET), }\AttributeTok{fill=}\NormalTok{OWN\_REALTY)) }\SpecialCharTok{+}
  \FunctionTok{geom\_bar}\NormalTok{(}\AttributeTok{color=}\StringTok{"black"}\NormalTok{, }\AttributeTok{position=}\StringTok{"dodge"}\NormalTok{) }\SpecialCharTok{+}
  \FunctionTok{labs}\NormalTok{(}\AttributeTok{x=}\StringTok{"FINANCIAL DIFFICULTY?"}\NormalTok{) }\SpecialCharTok{+}
  \FunctionTok{scale\_x\_discrete}\NormalTok{(}\AttributeTok{label=}\FunctionTok{c}\NormalTok{(}\StringTok{"False"}\NormalTok{, }\StringTok{"True"}\NormalTok{))}
\end{Highlighting}
\end{Shaded}

\includegraphics{Assig1_files/figure-latex/unnamed-chunk-13-1.pdf}

\begin{Shaded}
\begin{Highlighting}[]
\FunctionTok{ggplot}\NormalTok{(data\_capped2, }\FunctionTok{aes}\NormalTok{(}\AttributeTok{x=}\FunctionTok{factor}\NormalTok{(TARGET), }\AttributeTok{fill=}\NormalTok{OWN\_REALTY)) }\SpecialCharTok{+}
  \FunctionTok{geom\_bar}\NormalTok{(}\AttributeTok{color=}\StringTok{"black"}\NormalTok{, }\AttributeTok{position=}\StringTok{"fill"}\NormalTok{) }\SpecialCharTok{+}
  \FunctionTok{labs}\NormalTok{(}\AttributeTok{x=}\StringTok{"FINANCIAL DIFFICULTY?"}\NormalTok{) }\SpecialCharTok{+}
  \FunctionTok{scale\_x\_discrete}\NormalTok{(}\AttributeTok{label=}\FunctionTok{c}\NormalTok{(}\StringTok{"False"}\NormalTok{, }\StringTok{"True"}\NormalTok{))}
\end{Highlighting}
\end{Shaded}

\includegraphics{Assig1_files/figure-latex/unnamed-chunk-13-2.pdf}

\hypertarget{own_car}{%
\paragraph{OWN\_CAR}\label{own_car}}

\begin{Shaded}
\begin{Highlighting}[]
\FunctionTok{ggplot}\NormalTok{(data\_capped2, }\FunctionTok{aes}\NormalTok{(}\AttributeTok{x=}\FunctionTok{factor}\NormalTok{(TARGET), }\AttributeTok{fill=}\NormalTok{OWN\_CAR)) }\SpecialCharTok{+}
  \FunctionTok{geom\_bar}\NormalTok{(}\AttributeTok{color=}\StringTok{"black"}\NormalTok{, }\AttributeTok{position=}\StringTok{"dodge"}\NormalTok{) }\SpecialCharTok{+}
  \FunctionTok{labs}\NormalTok{(}\AttributeTok{x=}\StringTok{"FINANCIAL DIFFICULTY?"}\NormalTok{) }\SpecialCharTok{+}
  \FunctionTok{scale\_x\_discrete}\NormalTok{(}\AttributeTok{label=}\FunctionTok{c}\NormalTok{(}\StringTok{"False"}\NormalTok{, }\StringTok{"True"}\NormalTok{))}
\end{Highlighting}
\end{Shaded}

\includegraphics{Assig1_files/figure-latex/unnamed-chunk-14-1.pdf}

\begin{Shaded}
\begin{Highlighting}[]
\FunctionTok{ggplot}\NormalTok{(data\_capped2, }\FunctionTok{aes}\NormalTok{(}\AttributeTok{x=}\FunctionTok{factor}\NormalTok{(TARGET), }\AttributeTok{fill=}\NormalTok{OWN\_CAR)) }\SpecialCharTok{+}
  \FunctionTok{geom\_bar}\NormalTok{(}\AttributeTok{color=}\StringTok{"black"}\NormalTok{, }\AttributeTok{position=}\StringTok{"fill"}\NormalTok{) }\SpecialCharTok{+}
  \FunctionTok{labs}\NormalTok{(}\AttributeTok{x=}\StringTok{"FINANCIAL DIFFICULTY?"}\NormalTok{) }\SpecialCharTok{+}
  \FunctionTok{scale\_x\_discrete}\NormalTok{(}\AttributeTok{label=}\FunctionTok{c}\NormalTok{(}\StringTok{"False"}\NormalTok{, }\StringTok{"True"}\NormalTok{))}
\end{Highlighting}
\end{Shaded}

\includegraphics{Assig1_files/figure-latex/unnamed-chunk-14-2.pdf}

\hypertarget{response-part-1-2}{%
\paragraph{Response (part 1):}\label{response-part-1-2}}

For the two variables, OWN\_REALTY and OWN\_CAR, I do not see any
significant differences. Since there were a lot more data in the
``False'' column, I needed to normalize it by putting them into a
stacked bar graph. Once normalized, it was quite apparent that owning
realty or cars does not have any significant influence over whether or
not someone has had past financial struggles.

\hypertarget{name_education_type}{%
\paragraph{NAME\_EDUCATION\_TYPE}\label{name_education_type}}

\begin{Shaded}
\begin{Highlighting}[]
\FunctionTok{ggplot}\NormalTok{(data\_capped2, }\FunctionTok{aes}\NormalTok{(}\AttributeTok{x=}\NormalTok{NAME\_EDUCATION\_TYPE, }\AttributeTok{fill=}\FunctionTok{factor}\NormalTok{(TARGET))) }\SpecialCharTok{+}
  \FunctionTok{geom\_bar}\NormalTok{(}\AttributeTok{color=}\StringTok{"black"}\NormalTok{, }\AttributeTok{position=}\StringTok{"dodge"}\NormalTok{) }\SpecialCharTok{+}
  \FunctionTok{labs}\NormalTok{(}\AttributeTok{fill=}\StringTok{"FINANCIAL DIFFICULTY?"}\NormalTok{, }\AttributeTok{x=}\StringTok{"EDUCATION TYPE"}\NormalTok{) }\SpecialCharTok{+}
  \FunctionTok{scale\_fill\_discrete}\NormalTok{(}\AttributeTok{label=}\FunctionTok{c}\NormalTok{(}\StringTok{"False"}\NormalTok{, }\StringTok{"True"}\NormalTok{)) }\SpecialCharTok{+}
  \FunctionTok{scale\_x\_discrete}\NormalTok{(}\AttributeTok{label=}\FunctionTok{c}\NormalTok{(}\StringTok{"Academic}\SpecialCharTok{\textbackslash{}n}\StringTok{Degree"}\NormalTok{, }\StringTok{"Higher"}\NormalTok{, }\StringTok{"Incomplete}\SpecialCharTok{\textbackslash{}n}\StringTok{Higher"}\NormalTok{, }\StringTok{"Lower}\SpecialCharTok{\textbackslash{}n}\StringTok{Secondary"}\NormalTok{, }\StringTok{"Secondary"}\NormalTok{))}
\end{Highlighting}
\end{Shaded}

\includegraphics{Assig1_files/figure-latex/unnamed-chunk-15-1.pdf}

\begin{Shaded}
\begin{Highlighting}[]
\FunctionTok{ggplot}\NormalTok{(data\_capped2, }\FunctionTok{aes}\NormalTok{(}\AttributeTok{fill=}\FunctionTok{factor}\NormalTok{(TARGET), }\AttributeTok{x=}\NormalTok{NAME\_EDUCATION\_TYPE)) }\SpecialCharTok{+}
  \FunctionTok{geom\_bar}\NormalTok{(}\AttributeTok{color=}\StringTok{"black"}\NormalTok{, }\AttributeTok{position=}\StringTok{"fill"}\NormalTok{) }\SpecialCharTok{+}
  \FunctionTok{labs}\NormalTok{(}\AttributeTok{fill=}\StringTok{"FINANCIAL DIFFICULTY?"}\NormalTok{, }\AttributeTok{x=}\StringTok{"EDUCATION TYPE"}\NormalTok{) }\SpecialCharTok{+}
  \FunctionTok{scale\_fill\_discrete}\NormalTok{(}\AttributeTok{label=}\FunctionTok{c}\NormalTok{(}\StringTok{"False"}\NormalTok{, }\StringTok{"True"}\NormalTok{)) }\SpecialCharTok{+}
  \FunctionTok{scale\_x\_discrete}\NormalTok{(}\AttributeTok{label=}\FunctionTok{c}\NormalTok{(}\StringTok{"Academic}\SpecialCharTok{\textbackslash{}n}\StringTok{Degree"}\NormalTok{, }\StringTok{"Higher"}\NormalTok{, }\StringTok{"Incomplete}\SpecialCharTok{\textbackslash{}n}\StringTok{Higher"}\NormalTok{, }\StringTok{"Lower}\SpecialCharTok{\textbackslash{}n}\StringTok{Secondary"}\NormalTok{, }\StringTok{"Secondary"}\NormalTok{))}
\end{Highlighting}
\end{Shaded}

\includegraphics{Assig1_files/figure-latex/unnamed-chunk-15-2.pdf}

\hypertarget{response-part-2-1}{%
\paragraph{Response (part 2)}\label{response-part-2-1}}

Here, we finally see something happening! As we can see in the stacked
bar chart, education type does influence whether or not someone has a
history of financial difficulty.From the normalized chart, we can see
that as the level of formal education decreases, the occurrences of past
financial difficulty rises. While it's not a dramatic jump, this does
support this conclusion. (Note, ``Lower secondary'' is lower than
``Secondary'', so really it should be the last column.)

\hypertarget{question-4}{%
\subsubsection{Question 4}\label{question-4}}

\hypertarget{my-question}{%
\paragraph{My question:}\label{my-question}}

Provide three research questions that relate the variables: total
income, credit score, number of children, owning realty, owning a car,
and education level. Your questions should not be related to past
financial history and you don't have to use every variable.

\hypertarget{chatgpt-response}{%
\paragraph{ChatGPT response:}\label{chatgpt-response}}

Here are three focused research questions that incorporate some of the
specified variables:

\begin{enumerate}
\def\labelenumi{\arabic{enumi}.}
\item
  How does the level of education correlate with total income among
  individuals who own realty compared to those who do not?
\item
  What is the relationship between credit score and the likelihood of
  owning a car among families with varying numbers of children?
\item
  In households with children, how does total income influence the
  decision to invest in realty, and what role does education level play
  in this decision-making process?
\end{enumerate}

\hypertarget{question-5}{%
\subsubsection{Question 5}\label{question-5}}

\hypertarget{how-does-the-level-of-education-correlate-with-total-income-among-individuals-who-own-realty-compared-to-those-who-do-not}{%
\paragraph{1. How does the level of education correlate with total
income among individuals who own realty compared to those who do
not?}\label{how-does-the-level-of-education-correlate-with-total-income-among-individuals-who-own-realty-compared-to-those-who-do-not}}

\begin{Shaded}
\begin{Highlighting}[]
\FunctionTok{ggplot}\NormalTok{(data\_capped2, }\FunctionTok{aes}\NormalTok{(}\AttributeTok{x=}\NormalTok{NAME\_EDUCATION\_TYPE, }\AttributeTok{y=}\NormalTok{AMT\_INCOME\_TOTAL, }\AttributeTok{fill=}\NormalTok{OWN\_REALTY)) }\SpecialCharTok{+}
  \FunctionTok{geom\_boxplot}\NormalTok{() }\SpecialCharTok{+}
  \FunctionTok{scale\_x\_discrete}\NormalTok{(}\AttributeTok{label=}\FunctionTok{c}\NormalTok{(}\StringTok{"Academic}\SpecialCharTok{\textbackslash{}n}\StringTok{Degree"}\NormalTok{, }\StringTok{"Higher"}\NormalTok{, }\StringTok{"Incomplete}\SpecialCharTok{\textbackslash{}n}\StringTok{Higher"}\NormalTok{, }\StringTok{"Lower}\SpecialCharTok{\textbackslash{}n}\StringTok{Secondary"}\NormalTok{, }\StringTok{"Secondary"}\NormalTok{))}
\end{Highlighting}
\end{Shaded}

\includegraphics{Assig1_files/figure-latex/unnamed-chunk-16-1.pdf}

\hypertarget{response-2}{%
\paragraph{Response:}\label{response-2}}

\n Based off this visualization, level of education and total income are
positively correlated. The boxes represent income level, and they rise
on the chart as education rises (recall, LowerSecondary should be the
rightmost column). There are also no outliers at the top of the chart
for the ``Academic degree'' column, which makes it unique compared to
the other categories. This trend is the same among those who own realty
and those who do not. The only interaction between academic level and
realty ownership is at the ``Academic degree'' level, where among these
academics, those who own realty also tend to have higher total incomes
(or vice versa).

\end{document}
