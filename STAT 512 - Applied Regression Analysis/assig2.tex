% Options for packages loaded elsewhere
\PassOptionsToPackage{unicode}{hyperref}
\PassOptionsToPackage{hyphens}{url}
%
\documentclass[
]{article}
\usepackage{amsmath,amssymb}
\usepackage{iftex}
\ifPDFTeX
  \usepackage[T1]{fontenc}
  \usepackage[utf8]{inputenc}
  \usepackage{textcomp} % provide euro and other symbols
\else % if luatex or xetex
  \usepackage{unicode-math} % this also loads fontspec
  \defaultfontfeatures{Scale=MatchLowercase}
  \defaultfontfeatures[\rmfamily]{Ligatures=TeX,Scale=1}
\fi
\usepackage{lmodern}
\ifPDFTeX\else
  % xetex/luatex font selection
\fi
% Use upquote if available, for straight quotes in verbatim environments
\IfFileExists{upquote.sty}{\usepackage{upquote}}{}
\IfFileExists{microtype.sty}{% use microtype if available
  \usepackage[]{microtype}
  \UseMicrotypeSet[protrusion]{basicmath} % disable protrusion for tt fonts
}{}
\makeatletter
\@ifundefined{KOMAClassName}{% if non-KOMA class
  \IfFileExists{parskip.sty}{%
    \usepackage{parskip}
  }{% else
    \setlength{\parindent}{0pt}
    \setlength{\parskip}{6pt plus 2pt minus 1pt}}
}{% if KOMA class
  \KOMAoptions{parskip=half}}
\makeatother
\usepackage{xcolor}
\usepackage[margin=1in]{geometry}
\usepackage{color}
\usepackage{fancyvrb}
\newcommand{\VerbBar}{|}
\newcommand{\VERB}{\Verb[commandchars=\\\{\}]}
\DefineVerbatimEnvironment{Highlighting}{Verbatim}{commandchars=\\\{\}}
% Add ',fontsize=\small' for more characters per line
\usepackage{framed}
\definecolor{shadecolor}{RGB}{248,248,248}
\newenvironment{Shaded}{\begin{snugshade}}{\end{snugshade}}
\newcommand{\AlertTok}[1]{\textcolor[rgb]{0.94,0.16,0.16}{#1}}
\newcommand{\AnnotationTok}[1]{\textcolor[rgb]{0.56,0.35,0.01}{\textbf{\textit{#1}}}}
\newcommand{\AttributeTok}[1]{\textcolor[rgb]{0.13,0.29,0.53}{#1}}
\newcommand{\BaseNTok}[1]{\textcolor[rgb]{0.00,0.00,0.81}{#1}}
\newcommand{\BuiltInTok}[1]{#1}
\newcommand{\CharTok}[1]{\textcolor[rgb]{0.31,0.60,0.02}{#1}}
\newcommand{\CommentTok}[1]{\textcolor[rgb]{0.56,0.35,0.01}{\textit{#1}}}
\newcommand{\CommentVarTok}[1]{\textcolor[rgb]{0.56,0.35,0.01}{\textbf{\textit{#1}}}}
\newcommand{\ConstantTok}[1]{\textcolor[rgb]{0.56,0.35,0.01}{#1}}
\newcommand{\ControlFlowTok}[1]{\textcolor[rgb]{0.13,0.29,0.53}{\textbf{#1}}}
\newcommand{\DataTypeTok}[1]{\textcolor[rgb]{0.13,0.29,0.53}{#1}}
\newcommand{\DecValTok}[1]{\textcolor[rgb]{0.00,0.00,0.81}{#1}}
\newcommand{\DocumentationTok}[1]{\textcolor[rgb]{0.56,0.35,0.01}{\textbf{\textit{#1}}}}
\newcommand{\ErrorTok}[1]{\textcolor[rgb]{0.64,0.00,0.00}{\textbf{#1}}}
\newcommand{\ExtensionTok}[1]{#1}
\newcommand{\FloatTok}[1]{\textcolor[rgb]{0.00,0.00,0.81}{#1}}
\newcommand{\FunctionTok}[1]{\textcolor[rgb]{0.13,0.29,0.53}{\textbf{#1}}}
\newcommand{\ImportTok}[1]{#1}
\newcommand{\InformationTok}[1]{\textcolor[rgb]{0.56,0.35,0.01}{\textbf{\textit{#1}}}}
\newcommand{\KeywordTok}[1]{\textcolor[rgb]{0.13,0.29,0.53}{\textbf{#1}}}
\newcommand{\NormalTok}[1]{#1}
\newcommand{\OperatorTok}[1]{\textcolor[rgb]{0.81,0.36,0.00}{\textbf{#1}}}
\newcommand{\OtherTok}[1]{\textcolor[rgb]{0.56,0.35,0.01}{#1}}
\newcommand{\PreprocessorTok}[1]{\textcolor[rgb]{0.56,0.35,0.01}{\textit{#1}}}
\newcommand{\RegionMarkerTok}[1]{#1}
\newcommand{\SpecialCharTok}[1]{\textcolor[rgb]{0.81,0.36,0.00}{\textbf{#1}}}
\newcommand{\SpecialStringTok}[1]{\textcolor[rgb]{0.31,0.60,0.02}{#1}}
\newcommand{\StringTok}[1]{\textcolor[rgb]{0.31,0.60,0.02}{#1}}
\newcommand{\VariableTok}[1]{\textcolor[rgb]{0.00,0.00,0.00}{#1}}
\newcommand{\VerbatimStringTok}[1]{\textcolor[rgb]{0.31,0.60,0.02}{#1}}
\newcommand{\WarningTok}[1]{\textcolor[rgb]{0.56,0.35,0.01}{\textbf{\textit{#1}}}}
\usepackage{graphicx}
\makeatletter
\def\maxwidth{\ifdim\Gin@nat@width>\linewidth\linewidth\else\Gin@nat@width\fi}
\def\maxheight{\ifdim\Gin@nat@height>\textheight\textheight\else\Gin@nat@height\fi}
\makeatother
% Scale images if necessary, so that they will not overflow the page
% margins by default, and it is still possible to overwrite the defaults
% using explicit options in \includegraphics[width, height, ...]{}
\setkeys{Gin}{width=\maxwidth,height=\maxheight,keepaspectratio}
% Set default figure placement to htbp
\makeatletter
\def\fps@figure{htbp}
\makeatother
\setlength{\emergencystretch}{3em} % prevent overfull lines
\providecommand{\tightlist}{%
  \setlength{\itemsep}{0pt}\setlength{\parskip}{0pt}}
\setcounter{secnumdepth}{-\maxdimen} % remove section numbering
\ifLuaTeX
  \usepackage{selnolig}  % disable illegal ligatures
\fi
\IfFileExists{bookmark.sty}{\usepackage{bookmark}}{\usepackage{hyperref}}
\IfFileExists{xurl.sty}{\usepackage{xurl}}{} % add URL line breaks if available
\urlstyle{same}
\hypersetup{
  pdftitle={Assignment 2},
  pdfauthor={Chloe Moore},
  hidelinks,
  pdfcreator={LaTeX via pandoc}}

\title{Assignment 2}
\author{Chloe Moore}
\date{2024-10-16}

\begin{document}
\maketitle

\begin{Shaded}
\begin{Highlighting}[]
\FunctionTok{library}\NormalTok{(tidyverse)}
\FunctionTok{library}\NormalTok{(ggplot2)}
\end{Highlighting}
\end{Shaded}

\hypertarget{question-11.-exercise-2.20}{%
\subsubsection{Question 11. Exercise
2.20}\label{question-11.-exercise-2.20}}

\begin{Shaded}
\begin{Highlighting}[]
\NormalTok{nursing\_homeDF }\OtherTok{=} \FunctionTok{read.csv}\NormalTok{(}\StringTok{"WiscNursingHome.csv"}\NormalTok{)}
\NormalTok{nursing\_home2001 }\OtherTok{=} \FunctionTok{subset}\NormalTok{(nursing\_homeDF, CRYEAR }\SpecialCharTok{==} \DecValTok{2001}\NormalTok{) }\SpecialCharTok{\%\textgreater{}\%}
  \FunctionTok{mutate}\NormalTok{(}
    \AttributeTok{LOGTPY =} \FunctionTok{log}\NormalTok{(TPY),}
    \AttributeTok{LOGNUMBED =} \FunctionTok{log}\NormalTok{(NUMBED)}
\NormalTok{  )}
\end{Highlighting}
\end{Shaded}

\hypertarget{part-a}{%
\paragraph{Part a)}\label{part-a}}

\begin{Shaded}
\begin{Highlighting}[]
\FunctionTok{summary}\NormalTok{(nursing\_home2001}\SpecialCharTok{$}\NormalTok{LOGTPY) }\CommentTok{\# summary statistics LOGTPY}
\end{Highlighting}
\end{Shaded}

\begin{verbatim}
##    Min. 1st Qu.  Median    Mean 3rd Qu.    Max. 
##   2.511   4.041   4.396   4.368   4.700   6.088
\end{verbatim}

\begin{Shaded}
\begin{Highlighting}[]
\FunctionTok{summary}\NormalTok{(nursing\_home2001}\SpecialCharTok{$}\NormalTok{LOGNUMBED) }\CommentTok{\# summary statistics LOGNUMBED}
\end{Highlighting}
\end{Shaded}

\begin{verbatim}
##    Min. 1st Qu.  Median    Mean 3rd Qu.    Max. 
##   2.890   4.094   4.500   4.457   4.779   6.125
\end{verbatim}

\begin{Shaded}
\begin{Highlighting}[]
\FunctionTok{cor}\NormalTok{(nursing\_home2001}\SpecialCharTok{$}\NormalTok{LOGTPY, nursing\_home2001}\SpecialCharTok{$}\NormalTok{LOGNUMBED) }\CommentTok{\# correlation statistic}
\end{Highlighting}
\end{Shaded}

\begin{verbatim}
## [1] 0.9830461
\end{verbatim}

\begin{Shaded}
\begin{Highlighting}[]
\FunctionTok{ggplot}\NormalTok{(nursing\_home2001, }\FunctionTok{aes}\NormalTok{(}\AttributeTok{x=}\NormalTok{LOGNUMBED, }\AttributeTok{y=}\NormalTok{LOGTPY)) }\SpecialCharTok{+}
  \FunctionTok{geom\_point}\NormalTok{()}
\end{Highlighting}
\end{Shaded}

\includegraphics{assig2_files/figure-latex/unnamed-chunk-3-1.pdf}

\n Comments: The correlation statistic is very close to 1, which
indicates LOGTPY and LOGNUMBED are closely, positively correlated. The
scatter plot chart appears to show a linear trend.

\hypertarget{part-b}{%
\paragraph{Part b)}\label{part-b}}

\begin{Shaded}
\begin{Highlighting}[]
\NormalTok{fit1 }\OtherTok{=} \FunctionTok{lm}\NormalTok{(LOGTPY }\SpecialCharTok{\textasciitilde{}}\NormalTok{ LOGNUMBED, nursing\_home2001)}
\NormalTok{summary1 }\OtherTok{=} \FunctionTok{summary}\NormalTok{(fit1)}
\NormalTok{summary1}
\end{Highlighting}
\end{Shaded}

\begin{verbatim}
## 
## Call:
## lm(formula = LOGTPY ~ LOGNUMBED, data = nursing_home2001)
## 
## Residuals:
##      Min       1Q   Median       3Q      Max 
## -0.87482 -0.02201  0.01517  0.05316  0.28862 
## 
## Coefficients:
##             Estimate Std. Error t value Pr(>|t|)    
## (Intercept) -0.17469    0.04537   -3.85  0.00014 ***
## LOGNUMBED    1.01923    0.01012  100.73  < 2e-16 ***
## ---
## Signif. codes:  0 '***' 0.001 '**' 0.01 '*' 0.05 '.' 0.1 ' ' 1
## 
## Residual standard error: 0.09373 on 353 degrees of freedom
## Multiple R-squared:  0.9664, Adjusted R-squared:  0.9663 
## F-statistic: 1.015e+04 on 1 and 353 DF,  p-value: < 2.2e-16
\end{verbatim}

\n Comments: R\^{}2 = 0.9664 (adjusted R\^{}2=0.9663). Regression
coefficient LOGNUMBED = 1.01923. t-statistic = 100.73

\hypertarget{part-c}{%
\paragraph{Part c)}\label{part-c}}

\begin{Shaded}
\begin{Highlighting}[]
\NormalTok{b1 }\OtherTok{=}\NormalTok{ summary1}\SpecialCharTok{$}\NormalTok{coefficients[}\DecValTok{2}\NormalTok{, }\DecValTok{1}\NormalTok{]}
\NormalTok{seb1 }\OtherTok{=}\NormalTok{ summary1}\SpecialCharTok{$}\NormalTok{coefficients[}\DecValTok{2}\NormalTok{, }\DecValTok{2}\NormalTok{]}
\NormalTok{tcrit }\OtherTok{=} \DecValTok{2} \CommentTok{\# for t\_(353, 0.975)}
\end{Highlighting}
\end{Shaded}

\n (I) \(H_0:\beta_{1}=0\) vs \(H_a:\beta_{1}\ne0\). We reject the null
hypothesis because our t-statistic is greater than the critical t-value.
The p-value is less than our alpha level of 0.05, which supports this
rejection.

\begin{Shaded}
\begin{Highlighting}[]
\NormalTok{tstat }\OtherTok{=}\NormalTok{ b1}\SpecialCharTok{/}\NormalTok{seb1 }\CommentTok{\# 100.730193}
\NormalTok{pval }\OtherTok{=} \DecValTok{2}\SpecialCharTok{*}\NormalTok{(}\DecValTok{1}\SpecialCharTok{{-}}\FunctionTok{pt}\NormalTok{(}\FunctionTok{abs}\NormalTok{(tstat), }\DecValTok{353}\NormalTok{)) }\CommentTok{\# 0.0000000...}
\end{Highlighting}
\end{Shaded}

\n (II) \(H_0:\beta_{1}=1\) vs \(H_a:\beta_{1}\ne1\). We fail to reject
the null hypothesis because our t-statistic is less than the critical
t-value (however, it's very close!). The p-value is greater than our
alpha level of 0.05, which supports this conclusion.

\begin{Shaded}
\begin{Highlighting}[]
\NormalTok{tstat }\OtherTok{=}\NormalTok{ (b1}\DecValTok{{-}1}\NormalTok{)}\SpecialCharTok{/}\NormalTok{seb1 }\CommentTok{\# 1.900567}
\NormalTok{pval }\OtherTok{=} \DecValTok{2}\SpecialCharTok{*}\NormalTok{(}\DecValTok{1}\SpecialCharTok{{-}}\FunctionTok{pt}\NormalTok{(}\FunctionTok{abs}\NormalTok{(tstat), }\DecValTok{353}\NormalTok{)) }\CommentTok{\# 0.058173}
\end{Highlighting}
\end{Shaded}

\n (III) \(H_0:\beta_{1}=1\) vs \(H_a:\beta_{1}>1\). We might fail to
reject the null hypothesis because our t-statistic is less than the
critical t-value\ldots{} But the p-value is less than our alpha level of
0.05, so in this case, we will actually reject the null hypothesis.

\begin{Shaded}
\begin{Highlighting}[]
\NormalTok{tstat }\OtherTok{=}\NormalTok{ (b1}\DecValTok{{-}1}\NormalTok{)}\SpecialCharTok{/}\NormalTok{seb1 }\CommentTok{\# 1.900567}
\NormalTok{pval }\OtherTok{=}\NormalTok{ (}\DecValTok{1}\SpecialCharTok{{-}}\FunctionTok{pt}\NormalTok{(}\FunctionTok{abs}\NormalTok{(tstat), }\DecValTok{353}\NormalTok{)) }\CommentTok{\# 0.029087}
\end{Highlighting}
\end{Shaded}

\n (IV) \(H_0:\beta_{1}=1\) vs \(H_a:\beta_{1}<1\). We might fail to
reject the null hypothesis because our t-statistic is less than the
critical t-value\ldots{} And the p-value is greater than our alpha level
of 0.05, so we will indeed fail to reject the null hypothesis.

\begin{Shaded}
\begin{Highlighting}[]
\NormalTok{tstat }\OtherTok{=}\NormalTok{ (b1}\DecValTok{{-}1}\NormalTok{)}\SpecialCharTok{/}\NormalTok{seb1 }\CommentTok{\# 1.900567}
\NormalTok{pval }\OtherTok{=} \FunctionTok{pt}\NormalTok{(}\FunctionTok{abs}\NormalTok{(tstat), }\DecValTok{353}\NormalTok{) }\CommentTok{\# 0.970913}
\end{Highlighting}
\end{Shaded}


\end{document}
